\section{Presentation}
The GTG letters stand for Generic Trace Generator. The project comes from the 
authors of two INRIA projects, \textit{eztrace} that needed a good way to 
write traces and \textit{ViTE} that needed an easy way to generate traces. GTG
is a library that offers a simple and clear generic interface to write down
traces in various formats (currently Paj\'e and OTF).

\subsection{Authors}
The authorss of the project are :
\begin{itemize}
\item Fran\c ois Rue
\item Fran\c ois Trahay
\item Olivier Lagrasse
\item Johnny Jazeix
\item Kevin Coulomb
\end{itemize}
If you find some bugs or have some feedbacks, please contact us at
gtg-devel\@ inria.fr. We will try to fix it as soon as possible.

\subsection{Eztrace}
Eztrace is a powerful tool to dynamically generate traces of parallel 
executions. The current version enables to have information on MPI calls, on
pthread calls and use. The roadmap foresee to be able to trace any userdefine 
calls. The project is available at 
\url{https://gforge.inria.fr/projects/eztrace/}.

\subsection{ViTE}
The ViTE (Visual Trace Explorer) project is made to easily and simply view and 
analyze traces in various formats (Paj\'e, OTF plus TAU that will be added 
soon). The project is available at \url{http://vite.gforge.inria.fr/}.

