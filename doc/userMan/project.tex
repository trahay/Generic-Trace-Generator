\section{Presentation}
The \GTG{} letters stand for Generic Trace Generator. The project comes from the 
authors of two INRIA projects, \eztrace{} that needed a good way to 
write traces and \ViTE{} that needed an easy way to generate traces. \GTG{}
is a library that offers a simple and clear generic interface to write down
traces in various formats (currently \paje{} and \OTF).

\subsection{Authors}
The authors of the project are~:
\begin{itemize}
\item Fran\c cois Rue
\item Fran\c cois Trahay
\item Olivier Lagrasse
\item Johnny Jazeix
\item Kevin Coulomb
\end{itemize}
If you find some bugs or have some feedbacks, please contact us at \href{mailto:gtg-devel@inria.fr}{\nolinkurl{gtg-devel@inria.fr}}. We will try to fix it as soon as possible.

\subsection{\eztrace}
\eztrace{} is a powerful tool, under \href{http://www.gnu.org/licenses/gpl-2.0.html}{\textit{gpl2}} licence, to dynamically
generate traces of parallel executions. The current version enables to have
information on MPI calls, on pthread calls and use. The roadmap foresees to
be able to trace any user define calls. The project is available at 
\url{http://eztrace.gforge.inria.fr/}.

\subsection{\ViTE{}}
The \ViTE{} (Visual Trace Explorer) project, under \href{http://www.cecill.info/licences/Licence_CeCILL_V2-en.html}{\textit{CeCILL-A}} licence, is made to easily and simply view and 
analyze traces in various formats (\paje{}, \OTF{} plus \TAU{} that will be added 
soon). The project is available at \url{http://vite.gforge.inria.fr/}.

